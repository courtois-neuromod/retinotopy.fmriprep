\PassOptionsToPackage{unicode=true}{hyperref} % options for packages loaded elsewhere
\PassOptionsToPackage{hyphens}{url}
%
\documentclass[]{article}
\usepackage{lmodern}
\usepackage{amssymb,amsmath}
\usepackage{ifxetex,ifluatex}
\usepackage{fixltx2e} % provides \textsubscript
\ifnum 0\ifxetex 1\fi\ifluatex 1\fi=0 % if pdftex
  \usepackage[T1]{fontenc}
  \usepackage[utf8]{inputenc}
  \usepackage{textcomp} % provides euro and other symbols
\else % if luatex or xelatex
  \usepackage{unicode-math}
  \defaultfontfeatures{Ligatures=TeX,Scale=MatchLowercase}
\fi
% use upquote if available, for straight quotes in verbatim environments
\IfFileExists{upquote.sty}{\usepackage{upquote}}{}
% use microtype if available
\IfFileExists{microtype.sty}{%
\usepackage[]{microtype}
\UseMicrotypeSet[protrusion]{basicmath} % disable protrusion for tt fonts
}{}
\IfFileExists{parskip.sty}{%
\usepackage{parskip}
}{% else
\setlength{\parindent}{0pt}
\setlength{\parskip}{6pt plus 2pt minus 1pt}
}
\usepackage{hyperref}
\hypersetup{
            pdfborder={0 0 0},
            breaklinks=true}
\urlstyle{same}  % don't use monospace font for urls
\setlength{\emergencystretch}{3em}  % prevent overfull lines
\providecommand{\tightlist}{%
  \setlength{\itemsep}{0pt}\setlength{\parskip}{0pt}}
\setcounter{secnumdepth}{0}
% Redefines (sub)paragraphs to behave more like sections
\ifx\paragraph\undefined\else
\let\oldparagraph\paragraph
\renewcommand{\paragraph}[1]{\oldparagraph{#1}\mbox{}}
\fi
\ifx\subparagraph\undefined\else
\let\oldsubparagraph\subparagraph
\renewcommand{\subparagraph}[1]{\oldsubparagraph{#1}\mbox{}}
\fi

% set default figure placement to htbp
\makeatletter
\def\fps@figure{htbp}
\makeatother

\usepackage[]{natbib}
\bibliographystyle{plainnat}

\date{}

\begin{document}

Results included in this manuscript come from preprocessing performed
using \emph{fMRIPrep} 20.2.6 (\citet{fmriprep1}; \citet{fmriprep2};
RRID:SCR\_016216), which is based on \emph{Nipype} 1.7.0
(\citet{nipype1}; \citet{nipype2}; RRID:SCR\_002502).

\begin{description}
\item[Anatomical data preprocessing]
A total of 0 T1-weighted (T1w) images were found within the input BIDS
dataset. Anatomical preprocessing was reused from previously existing
derivative objects.
\item[Functional data preprocessing]
For each of the 1 BOLD runs found per subject (across all tasks and
sessions), the following preprocessing was performed. First, a reference
volume and its skull-stripped version were generated by aligning and
averaging 1 single-band references (SBRefs). A B0-nonuniformity map (or
\emph{fieldmap}) was estimated based on two (or more) echo-planar
imaging (EPI) references with opposing phase-encoding directions, with
\texttt{3dQwarp} \citet{afni} (AFNI 20160207). Based on the estimated
susceptibility distortion, a corrected EPI (echo-planar imaging)
reference was calculated for a more accurate co-registration with the
anatomical reference. The BOLD reference was then co-registered to the
T1w reference using \texttt{bbregister} (FreeSurfer) which implements
boundary-based registration \citep{bbr}. Co-registration was configured
with six degrees of freedom. Head-motion parameters with respect to the
BOLD reference (transformation matrices, and six corresponding rotation
and translation parameters) are estimated before any spatiotemporal
filtering using \texttt{mcflirt} \citep[FSL 5.0.9,][]{mcflirt}. BOLD
runs were slice-time corrected to 0.685s (0.5 of slice acquisition range
0s-1.37s) using \texttt{3dTshift} from AFNI 20160207
\citep[RRID:SCR\_005927]{afni}. First, a reference volume and its
skull-stripped version were generated using a custom methodology of
\emph{fMRIPrep}. The BOLD time-series were resampled onto the following
surfaces (FreeSurfer reconstruction nomenclature): \emph{fsaverage}. The
BOLD time-series (including slice-timing correction when applied) were
resampled onto their original, native space by applying a single,
composite transform to correct for head-motion and susceptibility
distortions. These resampled BOLD time-series will be referred to as
\emph{preprocessed BOLD in original space}, or just \emph{preprocessed
BOLD}. The BOLD time-series were resampled into standard space,
generating a \emph{preprocessed BOLD run in MNI152NLin2009cAsym space}.
First, a reference volume and its skull-stripped version were generated
using a custom methodology of \emph{fMRIPrep}. \emph{Grayordinates}
files \citep{hcppipelines} containing 91k samples were also generated
using the highest-resolution \texttt{fsaverage} as intermediate
standardized surface space. Several confounding time-series were
calculated based on the \emph{preprocessed BOLD}: framewise displacement
(FD), DVARS and three region-wise global signals. FD was computed using
two formulations following Power (absolute sum of relative motions,
\citet{power_fd_dvars}) and Jenkinson (relative root mean square
displacement between affines, \citet{mcflirt}). FD and DVARS are
calculated for each functional run, both using their implementations in
\emph{Nipype} \citep[following the definitions by][]{power_fd_dvars}.
The three global signals are extracted within the CSF, the WM, and the
whole-brain masks. Additionally, a set of physiological regressors were
extracted to allow for component-based noise correction
\citep[\emph{CompCor},][]{compcor}. Principal components are estimated
after high-pass filtering the \emph{preprocessed BOLD} time-series
(using a discrete cosine filter with 128s cut-off) for the two
\emph{CompCor} variants: temporal (tCompCor) and anatomical (aCompCor).
tCompCor components are then calculated from the top 2\% variable voxels
within the brain mask. For aCompCor, three probabilistic masks (CSF, WM
and combined CSF+WM) are generated in anatomical space. The
implementation differs from that of Behzadi et al.~in that instead of
eroding the masks by 2 pixels on BOLD space, the aCompCor masks are
subtracted a mask of pixels that likely contain a volume fraction of GM.
This mask is obtained by dilating a GM mask extracted from the
FreeSurfer's \emph{aseg} segmentation, and it ensures components are not
extracted from voxels containing a minimal fraction of GM. Finally,
these masks are resampled into BOLD space and binarized by thresholding
at 0.99 (as in the original implementation). Components are also
calculated separately within the WM and CSF masks. For each CompCor
decomposition, the \emph{k} components with the largest singular values
are retained, such that the retained components' time series are
sufficient to explain 50 percent of variance across the nuisance mask
(CSF, WM, combined, or temporal). The remaining components are dropped
from consideration. The head-motion estimates calculated in the
correction step were also placed within the corresponding confounds
file. The confound time series derived from head motion estimates and
global signals were expanded with the inclusion of temporal derivatives
and quadratic terms for each \citep{confounds_satterthwaite_2013}.
Frames that exceeded a threshold of 0.5 mm FD or 1.5 standardised DVARS
were annotated as motion outliers. All resamplings can be performed with
\emph{a single interpolation step} by composing all the pertinent
transformations (i.e.~head-motion transform matrices, susceptibility
distortion correction when available, and co-registrations to anatomical
and output spaces). Gridded (volumetric) resamplings were performed
using \texttt{antsApplyTransforms} (ANTs), configured with Lanczos
interpolation to minimize the smoothing effects of other kernels
\citep{lanczos}. Non-gridded (surface) resamplings were performed using
\texttt{mri\_vol2surf} (FreeSurfer).
\end{description}

Many internal operations of \emph{fMRIPrep} use \emph{Nilearn} 0.6.2
\citep[RRID:SCR\_001362]{nilearn}, mostly within the functional
processing workflow. For more details of the pipeline, see
\href{https://fmriprep.readthedocs.io/en/latest/workflows.html}{the
section corresponding to workflows in \emph{fMRIPrep}'s documentation}.

\hypertarget{copyright-waiver}{%
\subsubsection{Copyright Waiver}\label{copyright-waiver}}

The above boilerplate text was automatically generated by fMRIPrep with
the express intention that users should copy and paste this text into
their manuscripts \emph{unchanged}. It is released under the
\href{https://creativecommons.org/publicdomain/zero/1.0/}{CC0} license.

\hypertarget{references}{%
\subsubsection{References}\label{references}}

\bibliography{/usr/local/miniconda/lib/python3.7/site-packages/fmriprep/data/boilerplate.bib}

\end{document}
